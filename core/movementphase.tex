% !TEX root = ../main.tex

\chapter{Movement Phase}

\section{Overview}

During his Turn a player may move any or all of his Units up to
their maximum Movement distance. Once one Unit has completed
all of its movement the player selects another Unit and moves that
one, and so on until the player has moved all the Units he wishes to
move for the current Movement Phase.


\section{The Movement Phase}

The normal Movement of Infantry models is 6”. A model may
neither move into nor through a gap between friendly models, nor
terrain pieces, smaller than its own base size.
A model cannot be placed so that it is within base-to-base contact
with an enemy model, and must remain at least 1” apart during the

Movement Phase. All models in a Unit move at the speed of the
slowest model.

If one model in a Unit moves during this Phase, all members of the
Unit are considered to have moved.

\section{Movement Phase Summery}

The Movement Phase proceeds as this:

\begin{enumerate}
\item  Choose Unit\\
The player selects any Unit that has not yet moved this Turn.
\item Move Unit\\
The player moves any or all models in the Unit up to their
maximum movement limit.
\item Repeat
\end{enumerate}

Return to Step One or conclude the Movement Phase.

\section{Unit Coherency}
As discussed on page 4. If Unit Coherency is broken (usually due
to taking casualties), the models in the unit must be moved to
restore Unit Coherency in the next Movement Phase. Until they do
so the Unit may not shoot nor launch an assault.

If the Unit cannot move for some reason in its next Turn (e.g. Due
to being pinned by shooting), then they must move to restore Unit
Coherency as soon as they are able.

\section{Turning \& Facing}
As models are moved they can turn by any amount without
penalty, to the maximum distance they are able to cover. Infantry
can be turned to face their targets during the Shooting Phase and
are not penalized for their facing during the Movement Phase.

\section{Random \& Compulsory \\Movement}
Some Units are subject to random or compulsory movement. Most
commonly this is d6 inches or 2d6 inches and/or moving towards
the closest enemy.

Unless otherwise specified in special rules for the Unit, normal
penalties for moving through Difficult or Dangerous Terrain
always apply. A Unit using Random Movement slowed by Difficult
Terrain halves the distance rolled (rounding up) unless otherwise
specified.