% !TEX root = ../main.tex

\chapter{Vehicle Characteristics}

\section{Overview}
Vehicles have Characteristics that define how powerful they are in
a similar way that Non-Vehicle Models do. However, as Vehicles do
not fight in the same way their Characteristics are different.

Unlike with Non-Vehicle Models the ranges can exceed 0-10.
Models with a Zero Level Characteristic may not test this
Characteristic


\section{Vehicle Profile}
The Vehicle Characteristics Profile looks like this:

\begin{tblr}{
	colsep = 0pt,
	width = \linewidth,
	colspec = {X[3,l] X[1,c] X[1,c] X[1,c] X[1,c] },
	% columns = {3em, c},
	% column{1} = {10em, l}
	}
& & \SetCell[c=3]{c} {\bfseries \reflectbox{\textlnot} Armour \textlnot} \\
 & {\bfseries BS}  & {\bfseries Front}  & {\bfseries Side} & {\bfseries Rear} \\
Predator Annihilator & 4 & 13 & 11 & 10\\
\end{tblr}

\section{Walker Profile}
The Walker Characteristics Profile looks like this:

\begin{tblr}{
	colsep = 0pt,
	width = \linewidth,
	colspec = {X[3.5,l] X[1,c] X[1,c] X[1,c] X[1.5,c] X[1.5,c] X[1.5,c] X[1,c] X[1,c]}
	% colspec = {c c c c c c c},
	% columns = {1.75em, c},
	% column{1} = {10em, l},
	% column{4} = {3em, c},
	% column{5} = {3em, c},
	% column{6} = {3em, c}
	}
& & &  & \SetCell[c=3]{c} {\bfseries \reflectbox{\textlnot} Armour \textlnot} \\
 & {\bfseries WS}  & {\bfseries BS} & {\bfseries S} & {\bfseries Front} & {\bfseries Side} & {\bfseries Rear} & {\bfseries I} & {\bfseries A} \\
Dreadnaught & 4 & 4 & 7 & 12 & 12 & 10 & 4 & 2 \\
\end{tblr}


\section{Vehicle Characteristics \\ In Detail}
Here's a quick summary to what it stands for:

\subsection{Type}
Special rules for different Vehicle types can be found on page \pageref{sec:vehicletypes}.

\subsection{Armour}
Separate Armor values are given for the Front (F), Sides (S),
and Rear (R) of each Vehicle Model. The values range from 10 -
14 and are used according to the front, side, rear of the Vehicle
Model being attacked.

When hit by a shot or blow, roll 1d6 + Strength (of the blow or
shot). Compare this result against the Armor value of the facing
struck then determine the result according to this chart:


\begin{center}
\begin{tblr}{
	colspec = {c c},
	% columns = {3em, c},
	}
Score vs Armour & Type of Hit\\\hline
greater than & penetrating hit\\
equal & glancing hit\\
less than & no effect
\end{tblr}
\end{center}

For each Glancing Hit or Penetrating Hit roll on the
following table to determine what happens to the Vehicle.


\begin{tblr}{
	colsep = 0pt,
	width = \linewidth,
	colspec = {X[1,c] X[6,c] X[6,c] X[6,c]},
	rows = {m},
	rowsep = 5pt
	% columns = {10em,c},
	% column{1} = {1.75em,c}
}
d6 & Glancing Hit & Penetrating Hit & Ordiance P. Hit\\\hline
1 & Shaken & Stunned & Stunned\\
2 & Shaken & {Stunned + \\Gun Wrecked} & {Stunned + \\Gun Wrecked}\\
3 & Stunned & {Stunned + \\Immobilized} & {Stunned + \\Immobilized}\\
4 & Gun Wrecked & Destroyed & Destroyed\\
5 & Immobilized & Destroyed & Explodes\\
6 & Destroyed & Explodes & Annihilated\\
\end{tblr}

\subsection{(WS), (BS), (S), (I), (A)}
Performs the same function as with Non-Vehicle Models. See page
\pageref{sec:characteristics} for more details on each specific Characteristic.

\section{Unlisted}
Every Vehicle models have three more values that won't appear on
a Vehicle Characteristics Profile:

\subsection{Transport Capacity}
Indicates how many human-sized infantry can be carried by the
Vehicle Model.

\subsection{Access Points}
Indicates where non-vehicle models may board a Transport-type
Vehicle model.

\subsection{Fire Points}
If the Vehicle Model is not Open-Topped, Fire Points indicates how
many passengers can fire their weapons while inside the Vehicle
model when being transported.

