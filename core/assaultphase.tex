% !TEX root = ../main.tex

\chapter{The Assault Phase}

\section{Overview}

The Assault Phase is comprised of two parts, Charges and
Close Combat. The summary of both is as follows.

\section{Charging}
The Charge process can be broken down into four steps:

\countsubsections

\subsection{Choose Valid Unit}
The Unit must have at least one model within assaulting range
of a target that it can reach while avoiding obstructions such as
Impassable Terrain and/or other enemy Units they don't wish
to charge. The Unit can't have fired twice with Pistol(s), shot
with Rapid Fire and/or Heavy Weapon(s) nor Falling Back or
being Pinned.

\subsection{Declare a Charge}

If the Unit fired in the Shooting Phase, it must target the Unit it
shot at. The charging Unit may targets multiple enemy Units
but may only do so if it could reach them all without losing Unit
Coherency. One of the enemy Units must be designated as
the primary target, and the charging Unit must move to engage
that one first and then the others. If it selects multiple targets
and also shot in the Shooting Phase this turn, it must
designate that enemy Unit as the primary target.

\subsection{Make an Assault}
Start the charge by moving the model that is closest to the
target enemy Unit up to its Assault move distance. It must end
in base-to-base contact with an enemy model, otherwise the
charge fails. If the model succeed, then move each other
model in the charging Unit one by one in the same way.

Assault move is 6” by default, unless otherwise specified. A
Unit with mixed Assault moves will move at the lowest value of
all models in the unit. Rules for Difficult or Dangerous Terrain
apply as usual and can cause a charge to fail.

Any charging model must attempt to move into base-to-base
contact with an enemy model that has no other charging
models in base-to-base contact. If the model cannot do that,
then it must end within 2” of another friendly model that is in
base-to-base contact. If the model can't do that either, it must
attempt to maintain Unit Coherency otherwise the charge fails.

Charging models cannot move through friendly or enemy
models nor move within 1” of enemy models from Units they
are not charging, nor pass through gaps narrower than their
base diameter.

If a Unit should fails its charge it stays in place and doesn't
move. Otherwise it is now Locked in Close Combat.

\subsection{Repeat the Above or Conclude}
You have to stop when you have no more valid Unit to charge
with or you can also decide to stop at any time by yourself.
Once you're done, resolve the Close Combat process by
following the broken down instructions found on the right.

\stopcountsubsections

\section{Close Combat}

Units that are Locked in Close Combat cannot shoot or be shot
at. Blast markers that scatter and templates can land on and
affect Locked units. Casualties caused by errant blast markers
or templates do not cause Pinning or Ld Test though.
The Close Combat process itself can be broken down into
three steps:

\countsubsections

\subsection{Pick a Unit Locked in Close Combat}

Models in base-to-base contact with an enemy model, or
within 2” of a friendly model that is in base-to-base contact with
an enemy model ,are said to be Engaged. Only these
Engaged models (from both armies) can fight in Close Combat
and become casualties as a result of it.


\subsection{Determining Attack order}

All Engaged models from both armies fight in Close Combat.
Begin with the Engaged model(s) with the highest Initiative (I)
and work down in descending order through all other models
Engaged in the Close Combat. Models that have not attacked
yet but are killed by higher Initiative models cannot attack.
Models that are Charged while they are within Cover are
treated as Initiative 10, as are attackers using Frag or Plasma
grenades during their Charge. Cover only affects the first
round of combat. After the initial round of combat, models fight
at Initiative Order.

\subsection{Check for extra Attacks}

Engaged models that charged this Phase get a +1A that only
lasts for the first round of combat. This very bonus is canceled
if the charged Unit(s) is equipped with Photon Grenades.
Engaged models with 2 or more single handed weapons
(typically a pistol and close combat weapon, or 2 close combat
weapons) get a +1A that is not limited by round, Phase or Turn.

\subsection{Roll to Hit}

When rolling To Hit, compare the attacker's WS to the
defender's WS on the Chart below to determine the target
number to be rolled for a successful Hit. When models attack,
calculate a model's To Hit roll based on its individual Weapon
Skill.

\begin{tblr}{
	colsep = 0pt,
	width = \linewidth,
	colspec = {X[0.6,c] X[1,c] |  X[1,c] X[1,c] X[1,c] X[1,c] X[1,c] X[1,c] X[1,c] X[1,c] X[1,c] X[1,c]},
	rows = {m},
	rowsep = 3pt
	% columns = {10em,c},
	% column{1} = {1.75em,c}
}
\SetCell[c=12]{c} Defender's Weapon Skill\\
 &    &  1 & 2  & 3  & 4  & 5  & 6  & 7  & 8  & 9  & 10 \\\hline
\SetCell[r=10]{m} \rotatebox[origin=c]{90}{Attacker's Weapon Skill}
 &  1 & 4+ & 4+ & 5+ & 5+ & 5+ & 5+ & 5+ & 5+ & 5+ & 5+  \\
 &  2 & 3+ & 4+ & 4+ & 4+ & 5+ & 5+ & 5+ & 5+ & 5+ & 5+  \\
 &  3 & 3+ & 3+ & 4+ & 4+ & 4+ & 4+ & 5+ & 5+ & 5+ & 5+  \\
 &  4 & 3+ & 3+ & 3+ & 4+ & 4+ & 4+ & 4+ & 4+ & 5+ & 5+  \\
 &  5 & 3+ & 3+ & 3+ & 3+ & 4+ & 4+ & 4+ & 4+ & 4+ & 4+  \\
 &  6 & 3+ & 3+ & 3+ & 3+ & 3+ & 4+ & 4+ & 4+ & 4+ & 4+  \\
 &  7 & 3+ & 3+ & 3+ & 3+ & 3+ & 3+ & 4+ & 4+ & 4+ & 4+ \\
 &  8 & 3+ & 3+ & 3+ & 3+ & 3+ & 3+ & 3+ & 4+ & 4+ & 4+ \\
 &  9 & 3+ & 3+ & 3+ & 3+ & 3+ & 3+ & 3+ & 3+ & 4+ & 4+ \\
 & 10 & 3+ & 3+ & 3+ & 3+ & 3+ & 3+ & 3+ & 3+ & 3+ & 4+ \\
\end{tblr}

When determining To Hit rolls, use whatever is the majority of
WS among engaged models in the enemy Unit. Where there
is no majority, always use the lower of the two values.



\subsection{Roll to Wound}
For all successful Hits, compare the Weapon's Strength
against the target's Toughness according to the table below.
Remember to account for any bonuses from equipped
Weapon Types. The number indicated is the minimum face
value on a d6 needed for the Hit to cause a Wounding Hit.
Results of N mean the Hit has no effect.


\begin{tblr}{
	colsep = 0pt,
	width = \linewidth,
	colspec = {X[0.6,c] X[1,c] |  X[1,c] X[1,c] X[1,c] X[1,c] X[1,c] X[1,c] X[1,c] X[1,c] X[1,c] X[1,c]},
	rows = {m},
	rowsep = 3pt
	% columns = {10em,c},
	% column{1} = {1.75em,c}
}
\SetCell[c=12]{c} Defender's Toughness\\
 &    &  1 & 2  & 3  & 4  & 5  & 6  & 7  & 8  & 9  & 10 \\\hline
\SetCell[r=10]{m} \rotatebox[origin=c]{90}{Attacker's Strength}
 &  1 & 4+ & 5+ & 6+ & 6+ & -  & -  & -  & -  & -  & -  \\
 &  2 & 3+ & 4+ & 5+ & 6+ & 6+ & -  & -  & -  & -  & -  \\
 &  3 & 2+ & 3+ & 4+ & 5+ & 6+ & 6+ & -  & -  & -  & -  \\
 &  4 & 2+ & 2+ & 3+ & 4+ & 5+ & 6+ & 6+ & -  & -  & -  \\
 &  5 & 2+ & 2+ & 2+ & 3+ & 4+ & 5+ & 6+ & 6+ & -  & -  \\
 &  6 & 2+ & 2+ & 2+ & 2+ & 3+ & 4+ & 5+ & 6+ & 6+ & -  \\
 &  7 & 2+ & 2+ & 2+ & 2+ & 2+ & 3+ & 4+ & 5+ & 6+ & 6+ \\
 &  8 & 2+ & 2+ & 2+ & 2+ & 2+ & 2+ & 3+ & 4+ & 5+ & 6+ \\
 &  9 & 2+ & 2+ & 2+ & 2+ & 2+ & 2+ & 2+ & 3+ & 4+ & 5+ \\
 & 10 & 2+ & 2+ & 2+ & 2+ & 2+ & 2+ & 2+ & 2+ & 3+ & 4+ \\
\end{tblr}

When determining To Wound rolls, use whatever is the
majority of T among engaged models in the enemy unit.
Where there is no majority, always use the lower of the two
values.

\subsection{Wound Allocation}
After determining the number of Wounding Hits, make Armor
Saves for all Hits. When Allocating Wounding Hits, no single
model may be said to take all hits (unless it is the only valid
target). Once all Wounding Hits and Wounds have been dealt,
if excess damage was dealt it is wasted. If only one Sv is being
used, once made for all Wounding Hits Wounds may then be
allocated to any models the targeted unit's player chooses, so
long as there are no multi-Wound models that have already
taken a Wound. Wounds must be applied to them first, without
``spreading around'' the damage to their fellows.

Units with multi-wound models must remove whole multi-
wound models as casualties where possible. Wounds cannot
be spread around.

\subsubsection{Who Can Have A Wound Allocated?}
{\bfseries Shooting:} Only models in range of the weapon causing the
save, or in line of sight, or eligible to have a wound allocated.
{\bfseries Torrent of Fire:} When a unit suffers as many (or more)
wounding hits from the firing of a single enemy unit as it has
models, the shooting player can nominate one model in the
target unit that could be a casualty. This model must make a
save against one of the wounding hits.

The owning player can choose which wounding hit the model
saves against, and if the model has more than one save, can
choose which one. Then all other saves are taken as normal.
{\bfseries Close Combat:} Only Engaged models in close combat are
eligible to have wounds allocated to them.
Casualties must be removed in such a way that the unit's
coherency is maintained.

\subsubsection{If a Unit has Mixed Armor}
\begin{enumerate}
\item Count up models of each Sv value (only valid targets).
\item  Determine the majority Sv, in the case of a tie the worst
Sv is assumed to be the majority.
\item  Apply Wounding Hits to the majority Sv first. If there are
any remaining Wounding Hits, apply to any remaining
models.
\item  Incoming hits form “sets” where every model must have
a Wounding Hit allocated before the next set is allocated.
\end{enumerate}

Once determined, check if any have to contend with Armor
Piercing for any or all allocated Wounding Hits. If so, resolve
those first then roll saving throws.

\subsection{Saving Throws}
Each model gets only one Saving Throw, chosen from
whichever options they have available to them. For each failed
Saving Throw the model takes a Wound.

When Wounds taken = W Characteristic, remove the model.
Excess wounds are lost. Units of multi-wound models must
remove whole models where possible. Once all eligible
Engaged models have fought, then proceed to the next step.

\subsubsection{Instant Death}
When Weapon S Characteristic is 2x Target's T Characteristic,
multiple Wound models suffer Instant Death on a failed Saving
Throw and are immediately killed.

\subsubsection{Cover Saves}
Cover does not give any benefit when making Saving Throws
in a Close Combat as it is already accounted for in the
Determining Attack Order step.

\subsection{Determine Winner of Assault}
Total up the number of Wounds inflicted by each side. The side
that inflicted the most Wounds is the winner. Note that this
does not include excess Wounds as those as discarded. If it is
a draw, nothing happens, proceed to Step 9. If one side lost all
of its models in the assault, it is counted as a Massacre,
proceed to Step 8.

\subsection{Loser Checks Morale}
The loser of the Assault must make a Morale Test (see page
XX). If passed, proceed to Step 9. Otherwise go to Step 7.

\subsection{Falling Back \& Sweeping Advances}
Both the loser and winner roll 1D6+Initiative. If the loser's
result is greater than the winner’s they break off from the Close
Combat successfully and make a Fall Back move. Otherwise
the winner makes a Sweeping Advance, causing the loser’s
Unit to be removed immediately from the board as they are
immediately Massacred.

However, if the winning Unit is still Locked in Close Combat
with another enemy Unit, or if the Unit currently Falling Back
no longer has any model in base-to-base contact with the
winner, the winning Unit may not use a Sweeping Advance but
may still make a Consolidation move.

\subsection{Consolidation}

After winning Close Combat, the victorious Unit may move up
to 3” in any direction, or engage new enemy Unit(s) if any are
within range. This movement may not be used to embark on a
transport Vehicle and the consolidating Unit must maintains
Unit Coherency regardless of how it moves. A unit which
consolidates into a new Close Combat does not count as
Engaged (but is now Locked) until the next Assault Phase.

If the victor caused a Massacre they may move 1D6” instead.
Consolidation movement does not trigger Dangerous Terrain
tests, nor is it slowed by Difficult Terrain.


\subsection{Pile-in Moves}
\label{Pile-in-Moves}
At this step of the Assault Phase, models in Units that were
Locked but not themselves Engaged must move up to 6” in an
attempt to contact the same enemy their other members were
Engaged with. Both players must Pile-in, beginning with the
player whose turn it is currently.
This is done the same as moving during a Charge, but does
not trigger Terrain Tests (like Consolidation). If the results of
the Pile-in does not result in Engaged models both sides then
Consolidate, starting with the player whose turn it is.

\subsection{Repeat}
Return to Step 1 and resolve Close Combat for all Units that
have not yet done so.

\stopcountsubsections

