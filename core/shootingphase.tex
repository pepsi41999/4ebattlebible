% !TEX root = ../main.tex

\chapter{The Shooting Phase}

\section{Overview}

During the Shooting Phase any or all of a player's Units may
fire, but each Unit must complete shooting before moving on
to the next.

Every model in a Unit may shoot, but Non-vehicle models can
only fire one weapon each. Vehicle Models may be able to fire
more than one weapon per model depending on availability of
targets. Individual models within a Unit can choose not to
shoot.

The whole Unit has to fire its weaponry at a single opposing
Unit – you may not split fire between two or more target Units.
Once the Shooting Phase is complete the player moves on to
their Assault Phase.

\section{Moving \& Shooting}
If any part of a unit moved during the preceding Movement
Phase, the whole unit counts as having moved.


\section{Shooting Phase Summary}

The Shooting Phase can be broken down into three steps:

\begin{enumerate}
	\item Choose a valid Unit to shoot with.
The Unit must have at least one model with a Ranged
Weapon, a target within range and line of sight to shoot at and
neither Falling Back nor being Pinned.

\item Resolve the Shooting Process for said Unit.
By following the broken down instructions found further down
on this page and the next one.

\item Repeat the above or conclude.
You have to stop when you have no more valid Unit to shoot
with or you can also decide to stop at any time by yourself.

\end{enumerate}

\section{The Shooting Process}
The Shooting Process follows seven major steps:

\countsubsections

\subsection{Choose a valid Target}
\label{step:choosevalidtarget}
Select an enemy Unit for your shooting Unit to fire at. Your Unit
may only fire at the nearest enemy Unit unless

\begin{itemize}
\item The nearest enemy Unit is Falling Back.
\item The nearest enemy Unit is Locked in Close Combat.
\item The Shooting Unit passes at a Ld Test to target a different
Unit instead. This Ld Test must be taken even if the Unit
automatically passes these tests or don't need to take them.
\item The next nearest enemy Unit is Size 3, which may be
targeted instead without having to take a Ld Test. If there is
another Size 3 Unit farther away than this one, you must
pass a Ld Test if you want to target that one instead.
\end{itemize}

You may not measure range before choosing a target. If you
are unsure of what target is closest, the unit should take a Ld
Test, then you can determine what target is closest.


\subsection{Line of Sight (LOS) \& Range}
Check if the enemy Unit is within the listed Range of the Unit's
weapon(s), if not those attacks miss automatically. Get down
to eye level of the table and see if you can draw an imaginary
uninterrupted firing line from your Unit to the targeted Unit. If
not, the attack fails automatically. If both within range and LOS
the model is a valid target and you may proceed further.
Additionally:
\begin{itemize}
\item Infantry models from the shooting Unit's army don't block
LOS.
\item All Size 3 models block LOS, the exception being
Skimmers that are not turned into Wrecks or not
Immobilized.
\item  Models locked in close combat block Line of Sight up to
those model's Size.
\item  Individual models in a Unit must have LOS in order for
them to fire.
\item  Units further than 6” within Area Terrain may not be
targeted, nor may they shoot, unless they are taller than the
Area Terrain.
\item Units may shoot targets behind, or deeper than 6” into
Area Terrain if they have a higher elevation, such as from a
cliff or building.
\end{itemize}

\subsection{Roll to Hit}
For all models able to Shoot at the target after the previous
steps, roll a d6 per shot and compare to a target number equal
to 7 – BS. There is normally no such thing as an automatic hit,
and a roll of 1 always misses.

Roll all To-Hit dice together. If firing multiple different weapons,
roll them separately so to not confuse the Hits, or use dice of
a different color to represent the different weapons' shots.

\subsection{Roll to Wound}

For all successful Hits, compare the Weapon's Strength
against the target's Toughness according to the table below.
The number indicated is the minimum face value on a d6
needed for the Hit to cause a Wounding Hit. Results of N mean
the Hit has no effect.


\begin{tblr}{
	colsep = 0pt,
	width = \linewidth,
	colspec = {X[0.6,c] X[1,c] |  X[1,c] X[1,c] X[1,c] X[1,c] X[1,c] X[1,c] X[1,c] X[1,c] X[1,c] X[1,c]},
	rows = {m},
	rowsep = 3pt
	% columns = {10em,c},
	% column{1} = {1.75em,c}
}
\SetCell[c=12]{c} Defender's Toughness\\
 &    &  1 & 2  & 3  & 4  & 5  & 6  & 7  & 8  & 9  & 10 \\\hline
\SetCell[r=10]{m} \rotatebox[origin=c]{90}{Weapons's Strength}
 &  1 & 4+ & 5+ & 6+ & 6+ & -  & -  & -  & -  & -  & -  \\
 &  2 & 3+ & 4+ & 5+ & 6+ & 6+ & -  & -  & -  & -  & -  \\
 &  3 & 2+ & 3+ & 4+ & 5+ & 6+ & 6+ & -  & -  & -  & -  \\
 &  4 & 2+ & 2+ & 3+ & 4+ & 5+ & 6+ & 6+ & -  & -  & -  \\
 &  5 & 2+ & 2+ & 2+ & 3+ & 4+ & 5+ & 6+ & 6+ & -  & -  \\
 &  6 & 2+ & 2+ & 2+ & 2+ & 3+ & 4+ & 5+ & 6+ & 6+ & -  \\
 &  7 & 2+ & 2+ & 2+ & 2+ & 2+ & 3+ & 4+ & 5+ & 6+ & 6+ \\
 &  8 & 2+ & 2+ & 2+ & 2+ & 2+ & 2+ & 3+ & 4+ & 5+ & 6+ \\
 &  9 & 2+ & 2+ & 2+ & 2+ & 2+ & 2+ & 2+ & 3+ & 4+ & 5+ \\
 & 10 & 2+ & 2+ & 2+ & 2+ & 2+ & 2+ & 2+ & 2+ & 3+ & 4+ \\
\end{tblr}


When determining To Wound rolls, use whatever is the
majority of T in the enemy Unit. Where there is no majority,
always use the lower of the two values.


\subsection{Wound Allocation}
After determining the number of Wounding Hits, make Armor
Saves for all Hits. When Allocating Wounding Hits, no single
model may be said to take all hits (unless it is the only valid
target). Once all Wounding Hits and Wounds have been dealt,
if excess damage was dealt it is wasted. If only one Sv is being
used, once made for all Wounding Hits Wounds may then be
allocated to any models the targeted Unit/s player chooses, so
long as there are no multi-Wound models that have already
taken a Wound. Wounds must be applied to them first, without
“spreading around” the damage to their fellows.

Units with multi-wound models must remove whole multi-
wound models as casualties where possible. Wounds cannot
be spread around.

\subsubsection{Who Can Have A Wound Allocated?}
{\bfseries Shooting:} Only models in range of the weapon causing the
save, or in line of sight, or eligible to have a wound allocated.
Torrent of Fire: When a unit suffers as many (or more)
wounding hits from the firing of a single enemy unit as it has
models, the shooting player can nominate one model in the
target unit that could be a casualty. This model must make a
save against one of the wounding hits.

The owning player can choose which wounding hit the model
saves against, and if the model has more than one save, can
choose which one. Then all other saves are taken as normal.
Close Combat: Only Engaged models in close combat are
eligible to have wounds allocated to them.
Casualties must be removed in such a way that the unit's
coherency is maintained.

\subsubsection{If a Unit has Mixed Armor}
\begin{enumerate}
\item Count up models of each Sv value (only valid targets).
\item Determine the majority Sv, in the case of a tie the worst
Sv is assumed to be the majority.
\item Apply Wounding Hits to the majority Sv first. If there are
any remaining Wounding Hits, apply to any remaining
models.
\item Incoming hits form “sets” where every model must have
a Wounding Hit allocated before the next set is allocated.
\end{enumerate}

Once determined, check if any have to contend with Armor
Piercing for any or all allocated Wounding Hits. If so, resolve
those first then roll saving throws.

\subsection{Saving Throws}
Each model gets only one Saving Throw, chosen from
whichever options they have available to them. For each failed
Saving Throw the model takes a Wound.

When Wounds taken = W Characteristic, remove the model.

\subsubsection{Instant Death}
When Weapon S Characteristic is 2x Target’s T Characteristic,
multiple Wound models suffer Instant Death on a failed Saving
Throw and are immediately killed.

\subsubsection{Armor Piercing Weapons (AP)}
When the Armor Piercing (AP) value is equal to or lower than
the target’s Sv the armor is ineffective and the model gets no
Saving Throw. If the AP value is higher the target makes a
Saving Throw as normal.


\subsubsection{Invulnerable Saves}
Models with an Invulnerable Save in their Profile may make
this Saving Throw if AP negates their normal Sv.

\subsubsection{Cover Saves}
If a Unit has more models within a piece of Cover than without
the entire Unit has a Cover Save. Cover Saves ignore Weapon
AP.

\begin{tblr}{
	colsep = 0pt,
	width = \linewidth,
	colspec = {X[3,l] X[1,c] X[1.5,c]},
	% columns = {3em, c},
	% column{1} = {10em, l}
	}
Terrain Type & Save & Models Effected \\\hline 
Brushes, Fences & 6+ & Size 1-2 \\
Crates, Pipes, Partially within & 5+ & Size 1-3\\
Wrecks, Ruins, Trenches & 4+ & Size 1-3\\
Bunkers, Fortifications & 3+ & Size 1\\
\end{tblr}



\subsection{Morale Checks For Casualties}
Any Unit suffering at least 25\% casualties to shooting during a
single Shooting Phase must make a Morale Test (see page
\pageref{sec:morale}).

\subsection{Repeat}
Return to Step \ref{step:choosevalidtarget} and resolve Shooting for all Units that have
not yet done so.

\stopcountsubsections